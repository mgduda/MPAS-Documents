%--------------------------------------------------------------------------------------------
% Running the MPAS Non-hydrostatic Atmosphere Model
%--------------------------------------------------------------------------------------------

\chapter{Running the MPAS Non-hydrostatic Atmosphere Model}
\label{chap:running_mpas_a}

\setlength\LTleft{0.0in}

Given an SCVT mesh, this chapter describes the two main steps to running the MPAS-Atmosphere model: creating initial conditions and running the model itself.  This chapter makes use of two MPAS cores, {\tt init\_atmosphere} and {\tt atmosphere}, which are, respectively, used for initializing and running the non-hydrostatic atmospheric model.  Sections \ref{sec:atm_ideal_init} and \ref{sec:atm_real_init} of this chapter describe the creation of idealized and real-data non-hydrostatic initial condition files using the {\tt init\_atmosphere} core. Section \ref{sec:atm_model_run} describes the basic procedure of running the model itself.

Each section of this chapter follows a familiar pattern of compiling and executing MPAS model components, albeit using different cores depending on its intended use.  The compilation will create either an initialization or a model executable, which are named, respectively, {\tt init\_atmosphere\_model} and {\tt atmosphere\_model}.  In general, an executable is run with {\tt mpiexec} or {\tt mpirun}, for example:

\vspace{12pt}
{\tt > mpiexec -n 8 atmosphere\_model}
\vspace{12pt}


\noindent where {\tt 8} is the number of MPI tasks to be used.  In any case where {\tt n} $>$ 1, there must exist a corresponding graph decomposition file, e.g., {\tt graph.info.part.8}. For more on graph decomposition, see Section \ref{sec:metis}.  

\section{Creating idealized ICs}
\label{sec:atm_ideal_init}

There are several idealized test cases supported within the {\tt init\_atmosphere} model initialization core:

1 --- Jablonowski and Williamson baroclinic wave, no initial perturbation
\footnote{Jablonowski, C. and D.L. Williamson, 2006, A baroclinic instability test case for atmospheric model dynamical cores, {\em QJRMS}, 132, 2943-2975. doi:10.1256/qj.06.12.}

2 --- Jablonowski and Williamson baroclinic wave, with initial perturbation

3 --- Jablonowski and Williamson baroclinic wave, with normal-mode perturbation

4 --- squall line

5 --- super-cell

6 --- mountain wave\\
\\
Creating idealized initial conditions is fairly straightforward, as no external data are required and the starting date/time is irrelevant to building the {\tt init.nc} file that will be used to run the model.

The following steps summarize the creation of {\tt init.nc}:

\begin{itemize}
\item Include a {\tt grid.nc} file, which contains the SCVT mesh, in the working directory
\item If running with more than one MPI task, include a {\tt graph.info.part.*} file in the working directory (Section \ref{sec:metis})
\item Compile MPAS with the {\tt init\_atmosphere} core specified (Section \ref{compiling_MPAS})
\item Edit the {\tt namelist.init\_atmosphere} configuration file (described below)
\item Edit the {\tt streams.init\_atmosphere} I/O configuration file (described below)
\item Run {\tt init\_atmosphere\_model} to create the initial condition file, {\tt init.nc}
\end{itemize}

When the {\tt init\_atmosphere\_model} executable is built, a default initialization namelist, {\tt namelist.init\_atmosphere}, will have been created. A number of the namelist parameters found in {\tt namelist.init\_atmosphere} are irrelevant to creating idealized conditions and can be removed or ignored.  The following table outlines the namelist parameters that are required; comments are given on the right for certain key parameters, and formal explanations for all namelist parameters can be found in Appendix \ref{chap:init_atm_namelist}.


\begin{longtable}{p{3in}|p{3.25in}}

\&nhyd\_model\\
   config\_init\_case = 2                      & a number between 1 and 6 corresponding to the cases listed at the beginning of this section\\
   config\_start\_time = '0000-01-01\_00:00:00' & the starting time for the simulation\\
   config\_theta\_adv\_order = 3                     & advection order for theta \\
/\\
\\
\&dimensions\\
   config\_nvertlevels = 26                      & the number of vertical levels to be used in the model \\
/\\
\\
\&decomposition\\
   config\_block\_decomp\_file\_prefix = 'graph.info.part.' & if running in parallel, needs to match the grid decomposition file prefix \\
/\\

\end{longtable}

After editing the {\tt namelist.init\_atmosphere} namelist file, the name of the input SCVT grid file, as well as the name of the initial condition file to be created, must be set in the XML I/O configuration file, {\tt streams.init\_atmosphere}. For a detailed description of the format of the XML I/O configuration file, refer to Chapter \ref{chap:mpas_io}. Specifically, the {\tt filename\_template} attribute must be set to the name of the SCVT grid file in the {\tt "input"} stream definition, and the {\tt filename\_template} attribute must be set to name of the initial condition file to be created in the {\tt "output"} stream definition.


\section{Creating real-data ICs}
\label{sec:atm_real_init}

Creating real-data initial conditions is similar to that of the idealized case described in the previous section, but is more involved as it requires interpolation of static geographic data (e.g., topography, land cover, soil category, etc.), surface fields such as soil temperature and SST, and the atmospheric initial conditions valid at a specific date and time. The static datasets are the same as those used by the WRF model, and the surface fields and atmospheric initial conditions can be obtained from, e.g., NCEP's GFS data using the WRF pre-processing system (WPS).

Creating real-data initial conditions requires a single compilation of the {\tt init\_atmosphere} core, but the actual generation of the IC files will take place using three separate runs of {\tt init\_atmosphere}, where each of these runs is described individually in the following sub-sections.  While it is possible to condense the three real-data initialization steps into fewer executions, running each step separately will both improve clarity and, as will become apparent, save a significant amount of time when generating subsequent initial conditions, that is, when making initial conditions using the same mesh but different starting times.

The first of the three steps, described in Section \ref{sec:atm_real_static}, is the interpolation of static fields onto the mesh to create a {\tt static.nc} file.  This step cannot be run in parallel and takes considerably longer than the steps that follow, however, the fields being static, this step need only be run once for a particular mesh, regardless of the number of initial condition files that are ultimately created from the {\tt static.nc} output file.  Described in Section \ref{sec:atm_real_surface} is an optional step that creates a file {\tt surface.nc} containing surface data at regular intervals. This file is used in the case where the model run makes periodic external updates to surface fields (currently only sea-ice and SST).  Finally, Section \ref{sec:atm_real_met} describes the processing of the atmospheric initial conditions beginning with the {\tt static.nc} file created in Section \ref{sec:atm_real_static}.  Naturally, each of the initialization runs described in the three following sections will make use of a {\tt namelist.init\_atmosphere} namelist file, and as was the case for idealized initial conditions, the default {\tt namelist.init\_atmosphere} file in the MPAS directory may be used as a starting point.  Not every variable in this namelist is needed for any particular step, and therefore each section will elaborate only on the namelist variables that are immediately relevant.

\subsection{Static fields}
\label{sec:atm_real_static}

The generation of a file {\tt static.nc} requires a set of static geographic data.  A suitable dataset can be obtained from the WRF model's download page \\
 \url{http://www.mmm.ucar.edu/wrf/users/download/get\_source.html}.  These static data files should be downloaded to a directory, which will be specified the {\tt namelist.init\_atmosphere} file (described below) prior to running this interpolation step.  The result of this run will be the creation of a NetCDF file ({\tt static.nc}), which is used in the two steps, described in Sections \label{sec:atm_real_surface} and \label{sec:atm_real_met}, following this to create surface update files and dynamic initial conditions.  Note that {\tt static.nc} can be generated once and then used repeatedly to generate surface update and initial condition files for different start times.

The following steps summarize the creation of {\tt static.nc}:

\begin{itemize}
\item Download geographic data from the WRF download page (described above) 
\item Compile MPAS with the {\tt init\_atmosphere} core specified (Section \ref{compiling_MPAS})
\item Include a {\tt grid.nc} file in the working directory
\item Edit the {\tt namelist.init\_atmosphere} configuration file (described below)
\item Edit the {\tt streams.init\_atmosphere} I/O configuration file (described below)
\item Run {\tt init\_atmosphere\_model} {\em with only one MPI task specified} to create {\tt static.nc}
\end{itemize}
Note that it is critical for this step that the initialization core is run serially; afterward, however, the steps described in \ref{sec:atm_real_surface} and \ref{sec:atm_real_met} may be run with more than one MPI task.

\begin{longtable}{p{3.0in} |p{3.25in}}

\&nhyd\_model\\
   config\_init\_case       = 7                      & must be 7, the real-data initialization case \\
/\\
\\
\&dimensions                                         & the following two variables must be present now, though their values do not become significant until \S 3.2.3 \\
   config\_nvertlevels     = 1                      &  \\
   config\_nsoillevels     = 1                       &  \\
/\\
\\
\&data\_sources\\
   config\_geog\_data\_path  = '/WPS\_GEOG/'         & absolute path to static files obtained from the WRF download page\\
/\\
\\
\&preproc\_stages                                    & only the static\_interp stage should be enabled \\
   config\_static\_interp   = .true.                 & \\
   config\_vertical\_grid   = .false.                & \\
   config\_met\_interp      = .false.                & \\
   config\_input\_sst       = .false.                & \\
/\\

\end{longtable}

After editing the {\tt namelist.init\_atmosphere} namelist file, the name of the input SCVT grid file, as well as the name of the static file to be created, must be set in the XML I/O configuration file, {\tt streams.init\_atmosphere}. For a detailed description of the format of the XML I/O configuration file, refer to Chapter \ref{chap:mpas_io}. Specifically, the {\tt filename\_template} attribute must be set to the name of the SCVT grid file in the {\tt "input"} stream definition, and the {\tt filename\_template} attribute must be set to name of the static file to be created in the {\tt "output"} stream definition.


\subsection{Surface field updates}
\label{sec:atm_real_surface}

This step is optional --- it is required only if surface fields are to be periodically updated during the model run.  The surface data could originate from any number of sources, though the most straightforward way to obtain a dataset in the appropriate format is to process GRIB data (e.g., GFS GRIB data) with the {\em ungrib} program of the WRF model's pre-processing system (WPS).  Detailed instructions for building and running the WPS, and the process of generating intermediate data files from GFS data, can be found in Chapter 3 of the WRF User Guide: \url{http://www.mmm.ucar.edu/wrf/users/docs/user\_guide/users\_guide\_chap3.htm}.

The following steps summarize the creation of {\tt surface.nc}:

\begin{itemize}
\item Include surface data intermediate files in the working directory
\item Include a {\tt static.nc} file in the working directory (Section \ref{sec:atm_real_static})
\item If running in parallel, include a {\tt graph.info.part.*} in the working directory (Section \ref{sec:metis})
\item Edit the {\tt namelist.init\_atmosphere} configuration file (see below)
\item Edit the {\tt streams.init\_atmosphere} I/O configuration file (described below)
\item Run {\tt init\_atmosphere\_model} to create {\tt surface.nc}
\end{itemize}


\begin{longtable}{p{3.0in} |p{3.25in}}

\&nhyd\_model\\
   config\_init\_case       = 8                      & must be 8, the surface field initialization case \\
   config\_start\_time      = '2010-10-23\_00:00:00' & time to begin processing surface data \\
   config\_stop\_time       = '2010-10-30\_00:00:00' & time to end processing surface data \\
/\\
\\
\&data\_sources\\
   config\_sfc\_prefix      = 'SST'                  & the prefix of the intermediate data files containing SST and sea-ice \\
   config\_fg\_interval     = 86400                  & interval between intermediate files to use for SST and sea-ice \\
/\\
\\
\\
\&preproc\_stages                                    & only the config\_input\_sst step should be enabled \\
   config\_static\_interp   = .false.                & \\
   config\_vertical\_grid   = .false.                & \\
   config\_met\_interp      = .false.                & \\
   config\_input\_sst       = .true.                 & \\
   config\_frac\_seaice    = .true.                 & \\
/\\
\\
\&decomposition\\
   config\_block\_decomp\_file\_prefix = 'graph.info.part.' & if running in parallel, needs to match the grid decomposition file prefix \\
/\\

\end{longtable}

After editing the {\tt namelist.init\_atmosphere} namelist file, the name of the static file, as well as the name of the surface update file to be created, must be set in the XML I/O configuration file, {\tt streams.init\_atmosphere}. Specifically, the {\tt filename\_template} attribute must be set to the name of the static file in the {\tt "input"} stream definition, and the {\tt filename\_template} attribute must be set to name of the surface update file to be created in the {\tt "output"} stream definition.

\subsection{Vertical grid generation and initial field interpolation}
\label{sec:atm_real_met}

The final step for creating a real-data initial conditions file ({\tt init.nc}) is to generate a vertical grid, the parameters of which will be specified in the {\tt namelist.init\_atmosphere} file, and to obtain an initial conditions dataset and interpolate it onto the model grid. As stated previously, while initial conditions could ultimately be obtained from many different data sources, here we assume the use of intermediate data files obtained from GFS data using the WPS ungrib program.  Detailed instructions for building and running the WPS, and how to generate intermediate data files from GFS data, can be found in Chapter 3 of the WRF user guide: \\
\url{http://www.mmm.ucar.edu/wrf/users/docs/user\_guide/users\_guide\_chap3.htm}.

The following steps summarize the creation of {\tt init.nc}:

\begin{itemize}
\item Include a WPS intermediate data file in the working directory
\item Include the {\tt static.nc} file in the working directory (Section \ref{sec:atm_real_static})
\item If running in parallel, include a {\tt graph.info.part.*} file in the working directory (Section \ref{sec:metis})
\item Edit the {\tt namelist.init\_atmosphere} configuration file (described below)
\item Edit the {\tt streams.init\_atmosphere} I/O configuration file (described below)
\item Run {\tt init\_atmosphere\_model} to create {\tt init.nc}
\end{itemize}


\begin{longtable}{p{3.0in} |p{3.25in}}

\&nhyd\_model\\
   config\_init\_case       = 7                      & must be 7 \\
   config\_start\_time      = '2010-10-23\_00:00:00' & time to process first-guess data \\
   config\_theta\_adv\_order = 3                     & advection order for theta \\
/\\
\\
\&dimensions\\
   config\_nvertlevels     = 41                      & number of vertical levels to be used in MPAS \\
   config\_nsoillevels     = 4                       & number of soil layers to be used in MPAS \\
   config\_nfglevels       = 38                      & number of vertical levels in intermediate file \\
   config\_nfgsoillevels   = 4                       & number of soil layers in intermediate file \\
/\\
\\
\&data\_sources\\
   config\_met\_prefix      = 'FILE'                 & the prefix of the intermediate file to be used for initial conditions \\
/\\
\\
\&vertical\_grid\\
   config\_ztop            = 30000.0                 & model top height (m) \\
   config\_nsmterrain      = 2                       & number of smoothing passes for terrain \\
   config\_smooth\_surfaces = .false.                 & whether to smooth zeta surfaces \\
/\\
\\
\&preproc\_stages                                    & \\
   config\_static\_interp   = .false.                & \\
   config\_vertical\_grid   = .true.                 & only these three stages should be enabled \\
   config\_met\_interp      = .true.                 & \\
   config\_input\_sst       = .false.                & \\
/\\
\\
\&decomposition\\
   config\_block\_decomp\_file\_prefix = 'graph.info.part.' & if running in parallel, needs to match the grid decomposition file prefix \\
/\\
\end{longtable}

After editing the {\tt namelist.init\_atmosphere} namelist file, the name of the static file, as well as the name of the initial condition file to be created, must be set in the XML I/O configuration file, {\tt streams.init\_atmosphere}. Specifically, the {\tt filename\_template} attribute must be set to the name of the static file in the {\tt "input"} stream definition, and the {\tt filename\_template} attribute must be set to name of the initial condition file to be created in the {\tt "output"} stream definition.

\section{Running the model}
\label{sec:atm_model_run}

With the files {\tt init.nc} and, optionally, {\tt surface.nc}, generated as in the previous sections, we have completed the prerequisites to run the model.  The only step remaining before running the model itself is the configuration of {\tt namelist.atmosphere}.  When the {\em atmosphere} core is built, a default {\tt namelist.atmosphere} namelist file will be automatically generated; this namelist can serve as a starting point for any modifications made following the steps below. This section will discuss both running the model from a cold start and restarting the model from some point in a previous run.

The following steps summarize running the model:

\begin{itemize}
\item Include an initial condition NetCDF file (e.g., {\tt init.nc}) in the working directory(Section \ref{sec:atm_ideal_init}, Section \ref{sec:atm_real_init})
\item If using surface updates, include a surface NetCDF file (e.g., {\tt surface.nc}) in the working directory (Section \ref{sec:atm_real_surface})
\item If running in parallel, include a graph decomposition file in the working directory (Section \ref{sec:metis})
\item If the MPAS directory has not been cleaned since running initialization, run {\tt make clean} with the {\tt atmosphere} core specified
\item Compile MPAS with the {\tt atmosphere} core specified (Section \ref{compiling_MPAS})
\item Edit the default {\tt namelist.atmosphere} configuration file (described below)
\item Edit the {\tt streams.atmosphere} I/O configuration file (described below)
\item Run the {\tt atmosphere\_model} executable
\end{itemize}

Below is a list of variables in {\tt namelist.atmosphere} that pertain to model timestepping, explicit horizontal diffusion, and model restarts.  A number of namelist variables are not listed here (specifications for dynamical core configuration, physics parameters, etc.) and Appendix \ref{chap:atm_namelist} should be consulted for the purpose and acceptable values of these parameters.

\begin{longtable}{p{3.0in} |p{3.25in}}

\&nhyd\_model                                        & \\
   config\_dt = 450.0                                & the model timestep; an appropriate value must be chosen relative to the grid cell spacing \\
   config\_start\_time = '2010-10-23\_00:00:00'      & the model start time corresponding to {\tt init.nc}\\
   config\_run\_duration = '5\_00:00:00'             & the duration of the model run; for format rules, see Appendix \ref{chap:atm_namelist} \\
   config\_len\_disp = 120000.0            & the smallest cell-to-cell distance in the mesh, used for computing a dissipation length scale\\
/                                                    & \\
\\
\&decomposition                                            & \\
   config\_block\_decomp\_file\_prefix = 'graph.info.part.' & if running in parallel, must match the prefix of the graph decomposition file \\
/                                                    & \\
\\
\&restart                                            & \\
   config\_do\_restart = .false.                     & if true, will select the appropriate {\tt restart.nc} file generated from a previous run \\
/ 
\end{longtable}

When running the model from a cold start, {\tt config\_start\_time} should match the time that was used when creating {\tt init.nc}.

Configuration of model input and output is accomplished by editing the {\tt streams.atmosphere} file. The following streams exist by default in the atmosphere
core:

\begin{longtable}{|p{1.25in} |p{5.0in}|}
\hline
   input        & the stream used to read model initial conditions for cold-start simulations \\ \hline
   restart      & the stream used to periodically write restart files during model integration, and to read initial conditions when performing a restart model run \\ \hline
   output      & the stream responsible for writing model prognostic and diagnostic fields to history files \\ \hline
   diagnostics & the stream responsible for writing (mostly) 2-d diagnostic fields, typically at higher temporal frequency than the history files \\ \hline
   surface    & the stream used to read periodic updates of sea-ice and SST from a surface update file created as described in Section 
\ref{sec:atm_real_surface} \\ \hline
\end{longtable}

\noindent For more information on the options available in the XML I/O configuration file, users are referred to Chapter \ref{chap:mpas_io}.

During the course of a model run, restart files are created at an interval specified by the {\tt output\_interval} attribute in the definition of the {\tt "restart"} stream. Running the model from a restart file is similar to running the model from {\tt init.nc}.  The required changes are that {\tt config\_do\_restart} must be set to {\tt .true.} and {\tt config\_start\_time} must correspond to a restart file existing in the working directory.

