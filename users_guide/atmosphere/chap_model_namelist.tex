%--------------------------------------------------------------------------------------------
% Model Namelist Options
%--------------------------------------------------------------------------------------------

\chapter{Model Namelist Options}
\label{chap:atm_namelist}

This chapter summarizes the complete set of namelist options available when running the MPAS non-hydrostatic atmosphere model.
All date-time string specifications are of the form described at the beginning of Appendix \ref{chap:init_atm_namelist}.

\section{nhyd\_model}

{\small
\begin{longtable}{|p{1.75in} |p{4.5in}|}
 \hline   
   config\_dt & Model time step, in seconds \newline 
   {\em Default value: 600.0} \\ \hline

   config\_start\_time & Starting time for model run \newline 
   {\em Default value: '0000-01-01\_00:00:00'} \\ \hline

   config\_run\_duration & Length of model run \newline 
   {\em Default value: 'none'} \\ \hline

   config\_stop\_time  & Stopping time for model run \newline 
   {\em Default value: 'none'} \\ \hline

   config\_split\_dynamics\_transport & Whether to split integration of dynamics equations from scalar transport \newline 
   {\em Default value: .true.} \\ \hline
   
   config\_number\_of\_sub\_steps & Number of acoustic steps per large RK step \newline 
   {\em Default value: 2} \\ \hline

   config\_dynamics\_split\_steps & Number of full RK steps per timestep \newline 
   {\em Default value: 3} \\ \hline

   config\_h\_mom\_eddy\_visc2 & $\nabla^2$ eddy viscosity for horizontal diffusion of momentum \newline 
   {\em Default value: 0.0} \\ \hline

   config\_h\_mom\_eddy\_visc4 & $\nabla^4$ eddy hyper-viscosity for horizontal diffusion of momentum \newline 
   {\em Default value: 0.0} \\ \hline

   config\_v\_mom\_eddy\_visc2 & $\nabla^2$ eddy viscosity for vertical diffusion of momentum \newline 
   {\em Default value: 0.0} \\ \hline

   config\_h\_theta\_eddy\_visc2 & $\nabla^2$ eddy viscosity for horizontal diffusion of theta \newline 
   {\em Default value: 0.0} \\ \hline

   config\_h\_theta\_eddy\_visc4 & $\nabla^4$ eddy hyper-viscosity for horizontal diffusion of theta \newline 
   {\em Default value: 0.0} \\ \hline

   config\_v\_theta\_eddy\_visc2 & $\nabla^2$ eddy viscosity for vertical diffusion of theta \newline 
   {\em Default value: 0.0} \\ \hline

   config\_horiz\_mixing & Formulation of horizontal mixing: \newline
                                           `2d\_smagorinsky' = 2-d Smagorinsky formulation, \newline
                                           `2d\_fixed' = fixed eddy viscosity, \newline 
   {\em Default value: '2d\_smagorinsky'} \\ \hline

   config\_len\_disp & Horizontal length scale for Smagorinsky formulation of horizontal diffusion \newline 
   {\em Default value: 120000.0} \\ \hline

   config\_theta\_adv\_order & Horizontal advection order for theta \newline 
   {\em Default value: 3} \\ \hline

   config\_scalar\_adv\_order & Horizontal advection order for scalars \newline 
   {\em Default value: 3} \\ \hline

   config\_w\_adv\_order & Horizontal advection order for w \newline 
   {\em Default value: 3} \\ \hline

   config\_u\_vadv\_order & Vertical advection order for normal velocities (u) \newline 
   {\em Default value: 3} \\ \hline

   config\_w\_vadv\_order & Vertical advection order for w \newline 
   {\em Default value: 3} \\ \hline

   config\_theta\_vadv\_order & Vertical advection order for theta \newline 
   {\em Default value: 3} \\ \hline

   config\_scalar\_vadv\_order & Vertical advection order for scalars \newline 
   {\em Default value: 3} \\ \hline

   config\_coef\_3rd\_order & Upwinding coefficient in the 3rd order advection scheme. \hfill\break 0 $\le$ config\_coef\_3rd\_order $\le$ 1 \newline 
   {\em Default value: 0.25} \\ \hline
   
   config\_scalar\_advection & Whether to advect scalar fields \newline 
   {\em Default value: .true.} \\ \hline   

   config\_positive\_definite & Whether to enable positive-definite advection of scalars \newline 
   {\em Default value: .false.} \\ \hline

   config\_monotonic & Whether to enable monotonic limiter in scalar advection \newline 
   {\em Default value: .true.} \\ \hline

   config\_mix\_full & mix full $\theta$ and $u$ fields, or mix perturbation from initial state \newline 
   {\em Default value: .true.} \\ \hline   
      
   config\_epssm & Off-centering parameter for the vertically implicit acoustic \hfill \break integration, dimensionless \newline 
   {\em Default value: 0.1} \\ \hline

   config\_smdiv & 3D divergence damping coefficient, dimensionless. \newline 
   {\em Default value: 0.1} \\ \hline

   config\_h\_ScaleWithMesh & Scale eddy viscosities with mesh-density function for horizontal diffusion \newline 
   {\em Default value: .false.} \\ \hline
   
   config\_apvm\_upwinding & Amount of upwinding in APVM \newline 
   {\em Default value: 0.5} \\ \hline   

%   config\_newpx & Use new horizontal pressure gradient calculation \newline 
%   {\em Default value: .false.} \\ \hline
\end{longtable}
}

\section{damping}

{\small
\begin{longtable}{|p{2.0in} |p{4.25in}|}
 \hline
   config\_zd & Height MSL to begin w-damping profile \newline 
   {\em Default value: 22000.0} \\ \hline

   config\_xnutr & Maximum w-damping coefficient at model top \newline 
   {\em Default value: 0.0} \\ \hline
\end{longtable}
}

\section{io}

{\small
\begin{longtable}{|p{1.75in} |p{4.5in}|}
 \hline
   config\_pio\_num\_iotasks        & Number of I/O tasks to use \hfill\break 0 implies all MPI tasks are I/O tasks \newline 
   {\em Default value: 0} \\ \hline

   config\_pio\_stride        & The separation (stride) in MPI rank between I/O tasks \newline 
   {\em Default value: 1} \\ \hline
   
\end{longtable}
}

\section{decomposition}

{\small
\begin{longtable}{|p{2.0in} |p{4.25in}|}
 \hline
 
   config\_block\_decomp\_file\_prefix & Prefix of graph decomposition file, to be suffixed with the MPI task count \newline 
   {\em Default value: 'graph.info.part.'} \\ \hline

\end{longtable}
}

\section{restart}

{\small
\begin{longtable}{|p{2.0in} |p{4.25in}|}
 \hline
   config\_do\_restart & Whether this run of the model is a restart run \newline 
   {\em Default value: .false.} \\ \hline
 
   config\_do\_DAcycling & Whether to re-compute coupled fields $\theta_m$, $\tilde\rho$, $\rho u$,  etc. from uncoupled fields when restarting the model; used for cycling DA experiments that analyze uncoupled fields in restart files \newline 
   {\em Default value: .false.} \\ \hline   
\end{longtable}
}

\section{printout}

{\small
\begin{longtable}{|p{2.0in} |p{4.25in}|}
 \hline
   config\_print\_global\_minmax\_vel & Whether to print the global min/max of the horizontal normal velocity field at the end of each timestep \newline 
   {\em Default value: .true.} \\ \hline

   config\_print\_global\_minmax\_sca & Whether to print the global min/max of scalar fields at the end of each timestep \newline 
   {\em Default value: .false.} \\ \hline    
\end{longtable}
}
         
\section{physics}
\label{sec:physics_namelist}

{\small
\begin{longtable}{|p{2.0in} |p{4.25in}|}
 \hline

  config\_sst\_update & Logical used to update the Sea-Surface Temperatures (SSTs), and fractional sea-ice (if available). If set to true, SSTs are updated using the file config\_sfc\_update\_name. If set to false, SSTs remain fixed during the entire model run. \newline 
  {\em Default value: .false.} \\ \hline

  config\_sstdiurn\_update & If set to true, a diurnal cycle is applied to the SSTs. If set to false, SSTs remain constant during the entire day.\newline 
  {\em Default value: .false.} \\ \hline

  config\_deepsoiltemp\_update & If set to true, deep soil temperatures are slowly updated during the model run. If set to false, deep soil temperatures remain fixed during the entire run. \newline 
  {\em Default value: .false.} \\ \hline
  
  config\_radtlw\_interval & Temporal interval between calls to the parameterizations of long wave radiation, format {\em 'yyyy-mm-dd\_hh:mm:ss'}. \newline 
  {\em Default value: '00:30:00'} \\ \hline

  config\_radtsw\_interval & Temporal interval between calls to the parameterizations of short wave radiation, format {\em 'yyyy-mm-dd\_hh:mm:ss'}. \newline 
  {\em Default value: '00:30:00'} \\ \hline     

  config\_bucket\_update & Temporal interval between updates to restoring the accumulated rain and radiation fields below their respective bucket values, format {\em 'yyyy-mm-dd\_hh:mm:ss'}. \newline 
  {\em Default value: 'none'} \\ \hline      

  config\_physics\_suite & Physics suite: \newline
                                             `mesoscale\_reference' = a suite of physics tested for mesoscale resolutions, \newline
                                             `none' = no physics parameterizations, \newline
  {\em Default value: 'mesoscale\_reference'} \\ \hline

  config\_microp\_scheme & Cloud Microphysics schemes: \newline
                                             `off'  = no microphysics, \newline
                                             `suite'  = scheme determined by the choice of config\_physics\_suite, \newline
                                             `kessler' = Kessler, \newline 
                                             `wsm6       ' = WSM6 \newline
  {\em Default value: 'suite'} \\ \hline

  config\_convection\_scheme & Convection schemes: \newline
                                             `off' = no convection scheme, \newline
                                             `suite'  = scheme determined by the choice of config\_physics\_suite, \newline
                                             `kain\_fritsch' = Kain-Fritsch, \newline 
                                             `tiedtke' = Tiedtke \newline
  {\em Default value: 'suite'} \\ \hline

  config\_lsm\_scheme & Land-surface schemes: \newline
                                             `off' = no land surface option, \newline
                                             `suite'  = scheme determined by the choice of config\_physics\_suite, \newline
                                             `noah' = NOAH land-surface scheme \newline                                              
  {\em Default value: 'suite'} \\ \hline

  config\_pbl\_scheme & Planetary Boundary Layer schemes: \newline
                                             `off' = no boundary layer scheme, \newline
                                             `suite'  = scheme determined by the choice of config\_physics\_suite, \newline
                                             `ysu' = YSU PBL scheme \newline 
  {\em Default value: 'suite'} \\ \hline

  config\_radt\_cld\_scheme & Parameterization of the cloud fraction for the long wave and short wave radiation schemes: \newline
                                              `off' = if LW and SW radiation schemes are both `off', \newline
                                              `suite'  = scheme determined by the choice of config\_physics\_suite, \newline
                                              'cld\_incidence' = the cloud fraction is equal to 0 or 1, \newline
                                              'cld\_fraction' = the cloud fraction varies between 0 and 1,  as a function of the relative humidity \newline
   {\em Default value: 'suite'} \\ \hline

  config\_radt\_lw\_scheme & Long wave (LW) radiation schemes: \newline
                                             `off' = no long-wave radiation scheme, \newline
                                             `suite'  = scheme determined by the choice of config\_physics\_suite, \newline
                                             `cam\_lw' = CAM LW radiation scheme, \newline 
                                             `rrtmg\_lw' = RRTMG LW radiation scheme \newline
  {\em Default value: 'suite'} \\ \hline

  config\_radt\_sw\_scheme & Short wave (SW) radiation scheme: \newline
                                             `off' = no short-wave radiation scheme, \newline
                                             `suite'  = scheme determined by the choice of config\_physics\_suite, \newline
                                             `cam\_sw' = CAM SW radiation scheme, \newline
                                             `rrtmg\_sw' = RRTMG SW radiation scheme \newline                                                                                           
  {\em Default value: 'suite'} \\ \hline

  config\_sfclayer\_scheme &  Surface-layer schemes \newline
                                             `off' = no surface-layer scheme, \newline
                                             `suite'  = scheme determined by the choice of config\_physics\_suite, \newline
                                             `monin\_obukhov' = Monin-Obukhov scheme \newline                                              
  {\em Default value: 'suite'} \\ \hline
 
  config\_bucket\_radt &  Threshold value below which accumulated long wave and short wave radiation fields are restored to if config\_bucket\_update is different from \em 'none'. \newline
  {\em Default value: '1.0e9'} \\ \hline
  
  config\_bucket\_rainc &  Threshold value below which the accumulated convective precipitation is restored to if config\_bucket\_update is different from \em 'none'.\newline
  {\em Default value: '100.0'} \\ \hline
  
  config\_bucket\_rainnc &  Threshold value below which the accumulated grid-scale precipitation is restored to if config\_bucket\_update is different from \em 'none'. \newline
  {\em Default value: '100.0'} \\ \hline
  
\end{longtable}
}

