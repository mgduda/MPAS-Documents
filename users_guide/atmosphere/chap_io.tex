\chapter{Configuring Model Input and Output}
\label{chap:mpas_io}

\newlength{\immindent}
\settowidth{\immindent}{{\tt <immutable\_stream }}


\newlength{\mutindent}
\settowidth{\mutindent}{{\tt <stream }}

The reading and writing of model fields in MPAS is handled by user-configurable {\em streams}. 
A stream represents a fixed set of model fields, together with dimensions and attributes, that are
all written or read together to or from the same file or set of files. Each MPAS model core may define
its own set of default streams that it typically uses for reading initial conditions, for writing and reading
restart fields, and for writing additional model history fields. Besides these default streams, users may define
new streams to, e.g., write certain diagnostic fields at a higher temporal frequency than the usual model
history fields.

Streams are defined in XML configuration files that are created at build time for each model core. The name 
of this XML file is simply `streams.' suffixed with the name of the core. For example, the streams for 
the {\em atmosphere} core are defined in a file named `streams.atmosphere', and the streams for 
the {\em init\_atmosphere} core are defined in a file named `streams.init\_atmosphere'. An XML stream
file may further reference other text files that contain lists of the model fields that are read or written in
each of the streams defined in the XML stream file.

Changes to the XML stream configuration file will take effect the next time an MPAS core is run; there is no need
to re-compile after making modifications to the XML files. As described in the next section, it is therefore possible, e.g.,
to change the interval at which a stream is written, the template for the filenames associated with a stream, or the 
set of fields that are written to a stream, without the need to re-compile any code.

Two classes of streams exist in MPAS: {\em immutable} streams and {\em mutable} streams. Immutable streams
are those for which the set of fields that belong to the stream may not be modified at model run-time; however, it is
possible to modify the interval at which the stream is read or written, the filename template describing the files
containing the stream on disk, and several other parameters of the stream. In contrast, all aspects of mutable streams,
including the set of fields that belong to the stream, may be modified at run-time. The motivation for the creation of
two stream classes is the idea that an MPAS core may not function correctly if certain fields are not read in upon 
model start-up or written to restart files, and it is therefore not reasonable for users to modify this set of required fields 
at run-time. An MPAS core developer may choose to implement such streams as immutable streams. Since fields may
not be added to an immutable stream at run-time, new immutable streams may not be defined at run-time, and the only 
type of new stream that may be defined at run-time is the mutable stream type.

\section{XML stream configuration files}
\label{sec:xml_stream_format} 

The XML stream configuration file for an MPAS core always has a parent XML {\em element} named {\tt streams}, within which 
individual streams are defined:

\vspace{12pt}
\noindent {\tt <streams>} \newline
\newline
\hspace*{1cm}... one or more stream definitions ... \newline
\newline
\noindent {\tt </streams>} \newline
\vspace{12pt}

Immutable streams are defined with the {\tt immutable\_stream} element, and mutable streams are defined with the {\tt stream}
element: 

\vspace{12pt}
\noindent {\tt <immutable\_stream name="initial\_conditions"} \newline
\hspace*{\immindent}{\tt type="input"} \newline
\hspace*{\immindent}{\tt filename\_template="init.nc"} \newline
\hspace*{\immindent}{\tt input\_interval="initial\_only"} \newline
\hspace*{\immindent}{\tt />} \newline
\vspace{12pt} \newline
\noindent {\tt <stream name="history"} \newline
\hspace*{\mutindent}{\tt type="output"} \newline
\hspace*{\mutindent}{\tt filename\_template="output.\$Y-\$M-\$D\_\$h.\$m.\$s.nc"} \newline
\hspace*{\mutindent}{\tt output\_interval="6:00:00" />} \newline
\newline
\hspace*{1cm}... model fields belonging to this stream ... \newline
\newline
\noindent {\tt </stream>} \newline
\vspace{12pt}

As shown in the example stream definitions, above, both classes of stream have the following required attributes:

\begin{itemize}
\item {\tt name} --- A unique name used to refer to the stream.
\item {\tt type} --- The type of stream, either {\tt "input"}, {\tt "output"}, {\tt "input;output"}, or {\tt "none"}. A stream may be both an input
and an output stream (i.e., {\tt "input;output"}) if, for example, it is read once at model start-up to provide initial conditions and thereafter written 
periodically to provide model checkpoints. A stream may be defined as neither input nor output (i.e., {\tt "none"}) for the purposes of defining a 
set of fields for inclusion other streams. Note that, for immutable streams, the type attribute may not be changed at run-time.
\item {\tt filename\_template} --- The template for files that exist or will be created by the stream. The filename template may include any of the
following variables, which are expanded based on the simulated time at which files are first created.
\begin{itemize}
\item {\tt \$Y} --- Year
\item {\tt \$M} --- Month
\item {\tt \$D} --- Day of the month
\item {\tt \$d} --- Day of the year
\item {\tt \$h} --- Hour
\item {\tt \$m} --- Minute
\item {\tt \$s} --- Second
\end{itemize}
A filename template may include either a relative or an absolute path, in which case MPAS will attempt to create any directories 
in the path that do not exist, subject to filesystem permissions.
\item {\tt input\_interval} --- For streams that have type {\tt "input"} or {\tt "input;output"}, the interval, beginning at the model initial time,
at which the stream will be read. Possible values include a time interval specification in the format {\tt "YYYY-MM-DD\_hh:mm:ss"}; the value 
{\tt "initial\_only"}, which specifies that the stream is read only once at the model initial time; or the value {\tt "none"}, which specifies that 
the stream is not read during a model run.
\item {\tt output\_interval} --- For streams that have type {\tt "output"} or {\tt "input;output"}, the interval, beginning at the model initial time,
at which the stream will be written. Possible values include a time interval specification in the format {\tt "YYYY-MM-DD\_hh:mm:ss"}; the value 
{\tt "initial\_only"}, which specifies that the stream is written only once at the model initial time; or the value {\tt "none"}, which specifies that 
the stream is not written during a model run.
\end{itemize}

Finally, the set of fields that belong to a mutable stream may be specified with any combination of the following elements. Note that, for 
immutable streams, no fields are specified at run-time in the XML configuration file.

\begin{itemize}
\item {\tt var} --- Associates the specified variable with the stream. The variable may be any of those defined in an MPAS core's Registry.xml file, but may
not include individual constituent arrays from a var\_array.
\item {\tt var\_array} --- Associates all constituent variables in a var\_array, defined in an MPAS core's Registry.xml file, with the stream.
\item {\tt var\_struct} --- Associates all variables in a var\_struct, defined in an MPAS core's Registry.xml file, with the stream.
\item {\tt stream} --- Associates all explicitly associated fields in the specified stream with the stream; streams are not recursively included.
\item {\tt file} --- Associates all variables listed in the specified text file, with one field per line, with the stream.
\end{itemize}

\pagebreak

\section{Optional stream attributes}
\label{sec:optional_stream_atts} 

Besides the required attributes described in the preceding section, several additional, optional attributes may be added to
the definition of a stream.

\begin{itemize}
\item {\tt filename\_interval} --- The interval between the timestamps used in the construction of the names of files associated with
a stream. Possible values include a time interval specification in the format {\tt "YYYY-MM-DD\_hh:mm:ss"}; the value {\tt "none"}, indicating
that only one file containing all times is associated with the stream; the value {\tt "input\_interval"} that, for input type streams, indicates that
each time to be read from the stream will come from a unique file; or the value {\tt "output\_interval"} that, for output type streams, indicates 
that each time to be written to the stream will go to a unique file whose name is based on the timestamp of the data being written. The default
value is {\tt "input\_interval"} for input type streams and {\tt "output\_interval"} for output type streams. For streams of type {\tt "input;output"}, the
default filename interval is {\tt "input\_interval"} if the input interval is an interval (i.e., not {\tt "initial\_only"}), or {\tt "output\_interval"} otherwise. 
Refer to Section \ref{sec:filename_interval_example}
for an example of the use of the filename\_interval attribute.
\item {\tt reference\_time} --- A time that is an integral number of filename intervals from the timestamp of any file associated with the stream.
The default value is the start time of the model simulation. Refer to Section \ref{sec:reference_time_example} for an example of 
the use of the reference\_time attribute.
\item {\tt clobber\_mode} --- Specifies how a stream should handle attempts to write to a file that already exists. Possible values
for the mode include:
\begin{itemize}
\item {\tt "overwrite"} --- The stream is allowed to overwrite records in existing files and to append new records 
to existing files; records not explicitly written to are left untouched.
\item {\tt "truncate"} or {\tt "replace\_files"} --- The stream is allowed to overwrite existing files, which are first truncated 
to remove any existing records; this is equivalent to replacing any existing files with newly created files of the same name.
\item {\tt "append"} --- The stream is only allowed to append new records to existing files; existing records may not be overwritten.
\item {\tt "never\_modify"} --- The stream is not allowed to modify existing files in any way.
\end{itemize}
The default clobber mode for streams is {\tt "never\_modify"}. Refer to Section \ref{sec:append_example} for an example of the use of
the clobber\_mode attribute.
\item {\tt precision} --- The precision with which real-valued fields will be written or read in a stream. Possible values include 
{\tt "single"} for 4-byte real values, {\tt "double"} for 8-byte real values, or {\tt "native"}, which specifies that real-valued fields
will be written or read in whatever precision the MPAS core was compiled. The default value is {\tt "native"}. Refer to Section \ref{sec:filename_interval_example}
for an example of the use of the precision attribute.
\item {\tt packages} --- A list of packages attached to the stream. A stream will be active (i.e., read or written) only if at least one of 
the packages attached to it is active, or if no packages at all are attached. Package names are provided as a semi-colon-separated
list. Note that packages may only be defined in an MPAS core's Registry.xml file at build time. By default, no packages are attached
to a stream.
\end{itemize}

\section{Stream definition examples}
\label{sec:stream_example} 

This section provides several example streams that make use of the optional stream attributes described in Section \ref{sec:optional_stream_atts}.
All examples are of output streams, since it is more likely that a user will need to write additional fields than to read additional fields, which a model would need to be aware of; however, the concepts that are illustrated here translate directly to input streams as well.

\subsection{Example: a single-precision output stream with one month of data per file}
\label{sec:filename_interval_example}

In this example, the optional attribute specification {\tt filename\_interval="01-00\_00:00:00"} is added to force a new
output file to be created for the stream every month. Note that the general format for time interval specifications is {\tt YYYY-MM-DD\_hh:mm:ss},
where any leading terms can be omitted; in this case, the year part of the interval is omitted. To reduce the file size, the specification
{\tt precision="single"} is also added to force real-valued fields to be written as 4-byte floating-point values, rather than the default of 8 bytes.

\vspace{12pt}
\noindent {\tt <stream name="diagnostics"} \newline
\hspace*{\mutindent}{\tt type="output"} \newline
\hspace*{\mutindent}{\tt filename\_template="diagnostics.\$Y-\$M.nc"} \newline
\hspace*{\mutindent}{\tt filename\_interval="01-00\_00:00:00"} \newline
\hspace*{\mutindent}{\tt precision="single"} \newline
\hspace*{\mutindent}{\tt output\_interval="6:00:00" />} \newline
\newline
\hspace*{1cm}{\tt <var name="u10"/>} \newline
\hspace*{1cm}{\tt <var name="v10"/>} \newline
\hspace*{1cm}{\tt <var name="t2"/>} \newline
\hspace*{1cm}{\tt <var name="q2"/>} \newline
\newline
\noindent {\tt </stream>} \newline
\vspace{12pt}

The only fields that will be written to this stream are the hypothetical 10-m diagnosed wind components, the 2-m temperature, 
and the 2-m specific humidity variables. Also, note that the filename template only includes the year and month from the model 
valid time; this can be problematic when the simulation starts in the middle of a month, and a solution for this problem 
is illustrated in the example of Section \ref{sec:reference_time_example}.

\subsection{Example: appending records to existing output files}
\label{sec:append_example}

By default, streams will never modify existing files whose filenames match the name of a file that would otherwise be written
during the course of a simulation. However, when restarting a simulation that is expected to add more records to existing output 
files, it can be useful to instruct the MPAS I/O system to append these records, thereby modifying existing files. This may be 
accomplished with the {\tt clobber\_mode} attribute.

\vspace{12pt}
\noindent {\tt <stream name="diagnostics"} \newline
\hspace*{\mutindent}{\tt type="output"} \newline
\hspace*{\mutindent}{\tt filename\_template="diagnostics.\$Y-\$M.nc"} \newline
\hspace*{\mutindent}{\tt filename\_interval="01-00\_00:00:00"} \newline
\hspace*{\mutindent}{\tt precision="single"} \newline
\hspace*{\mutindent}{\tt clobber\_mode="append"} \newline
\hspace*{\mutindent}{\tt output\_interval="6:00:00" />} \newline
\newline
\hspace*{1cm}{\tt <var name="u10"/>} \newline
\hspace*{1cm}{\tt <var name="v10"/>} \newline
\hspace*{1cm}{\tt <var name="t2"/>} \newline
\hspace*{1cm}{\tt <var name="q2"/>} \newline
\newline
\noindent {\tt </stream>} \newline
\vspace{12pt}

In general, if MPAS were to attempt to write a record at a time that already existed in an output file, a clobber\_mode 
of `append' would not permit the write to take place, since this would modify existing data; in `append' mode, only new
records may be added. However, due to a peculiarity in the implementation of the `append' clobber mode, it may be 
possible for an output file to contain duplicate times. This can happen when the first record that is appended to 
an existing file has a timestamp not matching any in the file, after which, any record that is written --- regardless of 
whether its timestamp matches one already in the file --- will be appended to the end of the file. This situation may arise, 
for example, when restarting a model simulation with a shorter {\tt output\_interval} than was used in the original model simulation 
with an MPAS core that does not write the first output time for restart runs.

\subsection{Example: referencing filename intervals to a time other than the start time}
\label{sec:reference_time_example}

The example stream of the previous sections creates a new file each month during the simulation, and the filenames
contain only the year and month of the timestamp when the file was created. If a simulation begins at 00 UTC on the
first day of a month, then each file in the diagnostic stream will contain only output times that fall within the month
in the filename. However, if a simulation were to begin in the middle of a month --- for example, the month of June, 2014 --- 
the first diagnostics output file would have a filename of `diagnostics.2014-06.nc', but rather than containing only output
fields valid in June, it would contain all fields written between the middle of June and the middle of July, at which point
one month of simulation would have elapsed, and a new output file, `diagnostics.2014-07.nc', would be created.

In order to ensure that the file `diagnostics.2014-06.nc' contained only data from June 2014, the {\tt reference\_time}
attribute may be added such that the day, hour, minute, and second in the date and time represent the first day of the month
at 00 UTC. In this example, the year and month of the reference time are not important, since the purpose of the reference time
here is to describe to MPAS that the monthly filename interval begins (i.e., is referenced to) the first day of the month.

\vspace{12pt}
\noindent {\tt <stream name="diagnostics"} \newline
\hspace*{\mutindent}{\tt type="output"} \newline
\hspace*{\mutindent}{\tt filename\_template="diagnostics.\$Y-\$M.nc"} \newline
\hspace*{\mutindent}{\tt filename\_interval="01-00\_00:00:00"} \newline
\hspace*{\mutindent}{\tt reference\_time="2014-01-01\_00:00:00"} \newline
\hspace*{\mutindent}{\tt precision="single"} \newline
\hspace*{\mutindent}{\tt clobber\_mode="append"} \newline
\hspace*{\mutindent}{\tt output\_interval="6:00:00" />} \newline
\newline
\hspace*{1cm}{\tt <var name="u10"/>} \newline
\hspace*{1cm}{\tt <var name="v10"/>} \newline
\hspace*{1cm}{\tt <var name="t2"/>} \newline
\hspace*{1cm}{\tt <var name="q2"/>} \newline
\newline
\noindent {\tt </stream>} \newline
\vspace{12pt}

In general, the components of a timestamp, {\tt YYYY-MM-DD\_hh:mm:ss}, that are less significant than (i.e., to the right of) 
those contained in a filename template are important in a reference\_time. For example, with a {\tt filename\_template} that contained
only the year, the month component of the {\tt reference\_time} would become important to identify the month of the year on which
the yearly basis for filenames would begin.


