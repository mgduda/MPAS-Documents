\chapter{Configuring Model Input and Output}
\label{chap:mpas_io}

\newlength{\immindent}
\settowidth{\immindent}{{\tt <immutable\_stream }}


\newlength{\mutindent}
\settowidth{\mutindent}{{\tt <stream }}

The reading and writing of model fields in MPAS is handled by user-configurable {\em streams}. 
A stream represents a fixed set of model fields, together with dimensions and attributes, that are
all written or read together to or from the same file or set of files. Each MPAS model core may define
its own set of default streams that it typically uses for reading initial conditions, for writing and reading
restart fields, and for writing additional model history fields. Besides these default streams, users may define
new streams to, e.g., write certain diagnostic fields at a higher temporal frequency than the usual model
history fields.

Streams are defined in XML configuration files, which are created at build time for each model core. The name 
of this XML file is simply `streams.' suffixed with the name of the core. For example, the streams for 
the {\em atmosphere} core are defined in a file named `streams.atmosphere', and the streams for 
the {\em init\_atmosphere} core are defined in a file named `streams.init\_atmosphere'. An XML stream
file may further reference other text files that contain lists of the model fields that are read or written in
each of the streams defined in the XML stream file.

Changes to the XML stream configuration file will take effect the next time an MPAS core is run; there is no need
to re-compile after making modifications to the XML files. As described in the next section, it is therefore possible, e.g.,
to change the interval at which a stream is written, the template for the filenames associated with a stream, or the 
set of fields that are written to a stream, without the need to re-compile any code.

Two classes of streams exist in MPAS: {\em immutable} streams and {\em mutable} streams. Immutable streams
are those for which the set of fields that belong to the stream may not be modified at model run-time; however, it is
possible to modify the interval at which the stream is read or written, the filename template describing the files
containing the stream on disk, and several other parameters of the stream. In contrast, all aspects of mutable streams,
including the set of fields that belong to the stream, may be modified at run-time. The motivation for the creation of
two stream classes is the idea that an MPAS core may not function correctly if certain fields are not read in upon 
model start-up, and it is therefore not reasonable for users to modify this set of required fields at run-time. Consequently,
new immutable streams may not be defined at run-time; the only type of new stream that may be defined at run-time
is the mutable stream type.

\section{XML stream configuration files}
\label{sec:xml_stream_format} 

The XML stream configuration file for an MPAS core always has a parent XML {\em element} named {\tt streams}, within which 
individual streams are defined:

\vspace{12pt}
\noindent {\tt <streams>} \newline
... one or more stream definitions ... \newline
\noindent {\tt </streams>} \newline
\vspace{12pt}

Immutable streams are defined with the {\tt immutable\_stream} element, and mutable streams are defined with the {\tt stream}
element: 

\vspace{12pt}
\noindent {\tt <immutable\_stream name="initial\_conditions"} \newline
\hspace*{\immindent}{\tt type="input"} \newline
\hspace*{\immindent}{\tt filename\_template="init.nc"} \newline
\hspace*{\immindent}{\tt input\_interval="initial\_only"} \newline
\hspace*{\immindent}{\tt />} \newline
\vspace{12pt} \newline
\noindent {\tt <stream name="history"} \newline
\hspace*{\mutindent}{\tt type="output"} \newline
\hspace*{\mutindent}{\tt filename\_template="output.\$Y-\$M-\$D\_\$h.\$m.\$s.nc"} \newline
\hspace*{\mutindent}{\tt output\_interval="6:00:00"} \newline
... model fields belonging to this stream ... \newline
\noindent {\tt </stream>} \newline
\vspace{12pt}

As shown in the example stream definitions, above, both classes of stream have the following required attributes:

\begin{itemize}
\item {\tt name} --- A unique name used to reference the stream
\item {\tt type} --- The type of stream, either {\tt "input"}, {\tt "output"}, {\tt "input;output"}, or {\tt "none"}. A stream may be both an input
and and output stream (i.e., {\tt "input;output"}) if, for example, it is read once at model start-up to provide initial conditions and thereafter written 
periodically to provide model checkpoints. A stream may be defined as neither input nor output (i.e., {\tt "none"}) for the purposes of defining a 
set of fields for inclusion other streams.
\item {\tt filename\_template} --- The template for files that exist or will be created by the stream. The filename template may include any of the
following variables, which are expanded based on the simulated time at which files are first created.
\begin{itemize}
\item {\tt \$Y} --- Year
\item {\tt \$M} --- Month
\item {\tt \$D} --- Day of the month
\item {\tt \$d} --- Day of the year
\item {\tt \$h} --- Hour
\item {\tt \$m} --- Minute
\item {\tt \$s} --- Second
\end{itemize}
\item {\tt input\_interval} --- For streams that have type {\tt "input"} or {\tt "input;output"}, the interval, beginning at the model initial time
at which the stream will be read. Possible values include a time interval specification in the format {\tt "YYYY-MM-DD\_hh:mm:ss"}; the value 
{\tt "initial\_only"}, which specifies that the stream is read only once at the model initial time; or the value {\tt "none"}, which specifies that 
the stream is not read during a model run.
\item {\tt output\_interval} --- For streams that have type {\tt "output"} or {\tt "input;output"}, the interval, beginning at the model initial time
at which the stream will be written. Possible values include a time interval specification in the format {\tt "YYYY-MM-DD\_hh:mm:ss"}; the value 
{\tt "initial\_only"}, which specifies that the stream is written only once at the model initial time; or the value {\tt "none"}, which specifies that 
the stream is not written during a model run.
\end{itemize}

Finally, the set of fields that belong to a mutable stream may be specified with any combination of the following elements. Note that, for 
immutable streams, no fields are specified at run-time in the XML configuration file.

\begin{itemize}
\item {\tt var} --- Associates the specified variable with the stream.
\item {\tt var\_array} --- Associates the constituent variables in a var\_array, defined in an MPAS core's Registry.xml file, with the stream.
\item {\tt var\_struct} --- Associates all variables in a var\_struct, defined in an MPAS core's Registry.xml file, with the stream.
\item {\tt stream} --- Associates all explicitly associated fields in the specified stream with the stream; streams are not recursively included.
\item {\tt file} --- Associates all variables listed in the specified text file with the stream.
\end{itemize}

\section{Optional stream attributes}
\label{sec:optional_stream_atts} 

Besides the required attributes described in the preceding section, several additional, optional attributes may be added to
the definition of a stream.

\begin{itemize}
\item {\tt filename\_interval} --- The interval between the timestamps used in the construction of the names of files associated with
a stream. Possible values include a time interval specification in the format {\tt "YYYY-MM-DD\_hh:mm:ss"}; the value {\tt "none"}, indicating
that only one file containing all times is associated with the stream; the value {\tt "input\_interval"} that, for input type streams, indicates that
each time to be read from the stream will come from a unique file; or the value {\tt "output\_interval"} that, for output type streams, indicates 
that each time to be written to the stream will go to a unique file whose name is based on the timestamp of the data being written. The default
value is {\tt "input\_interval"} for input type streams and {\tt "output\_interval"} for output type streams.
\item {\tt reference\_time} --- A time that is an integral number of filename intervals from the timestamp of any file associated with the stream.
The default value is the start time of the model run.
\item {\tt clobber\_mode} --- Specifies how a stream should handle attempts to write to a file that already exists. Possible values
for the mode include:
\begin{itemize}
\item {\tt "overwrite"} --- The stream is allowed to overwrite records in existing files and to append new records 
to existing files; records not explicitly written to are left untouched.
\item {\tt "truncate"} or {\tt "replace\_files"} --- The stream is allowed to overwrite existing files, which are first truncated 
to remove any existing records; this is equivalent to replacing any existing files with newly created files of the same name.
\item {\tt "append"} --- The stream is only allowed to append new records to existing files; existing records may not be overwritten.
\item {\tt "never\_modify"} --- The stream is not allowed to modify existing files in any way.
\end{itemize}
The default clobber mode for streams is {\tt "never\_modify"}.
\item {\tt precision} --- The precision with which real-valued fields will be written or read in a stream. Possible values include 
{\tt "single"} for 4-byte real values, {\tt "double"} for 8-byte real values, or {\tt "native"}, which specifies that real-valued fields
will be written or read in whatever precision the MPAS core was compiled. The default value is {\tt "native"}.
\item {\tt packages} --- A list of packages attached to the stream. A stream will be active (i.e., read or written) only if at least one of 
the packages attached to it is active, or if no packages at all are attached. Package names are provided as a semi-colon-separated
list. Note that packages may only be defined in an MPAS core's Registry.xml file at build time. By default, no packages are attached
to a stream.
\end{itemize}

\section{Example: Defining a new output stream}
\label{sec:stream_example} 

For quasi-uniform meshes, very little preparation is actually needed, and
generally, one only needs to prepare mesh decomposition files --- files that
describe the decomposition of the SCVT mesh across processors --- when running
MPAS-A using multiple MPI tasks. The procedure for creating these mesh
decomposition files is described in the first section. 
