\chapter{MPAS-Atmosphere Quick Start Guide}
\label{chap:quick_start}

This chapter provides MPAS-Atmosphere users with a quick start description of
how to build and run the model. It is meant merely as a brief overview of the
process, while the more detailed descriptions of each step are provided in later
chapters.

In general, the build process follows the following steps.

\begin{enumerate}
	\item Build or locate an implementation of MPI (MPICH, OpenMPI, MVAPICH2, etc; Section \ref{build_prerequisites}).
	\item Build the serial NetCDF library (Section \ref{serial_netcdf}).
	\item Build the Parallel-NetCDF library (Section \ref{parallel_netcdf}).
	\item Build the Parallel I/O library (Section \ref{pio_build}).
	\item {\bf Optionally,} build the METIS package (Section \ref{sec:metis}).
	\item Obtain the MPAS source code.
	\item Build the {\em init\_atmosphere} and {\em atmosphere} cores (Section \ref{compiling_MPAS}).
\end{enumerate}

After completing these steps, executable files named {\tt init\_atmosphere\_model} and
{\tt atmosphere\_model} should have been created in the top-level MPAS directory. Once
both executables have been created, a complete model run can be completed with the following steps:

\begin{enumerate}
	\item Create a run directory.
	\item Link the {\tt init\_atmosphere\_model} and {\tt atmosphere\_model} executables to the run directory.
	\item Copy the {\tt namelist.*}, {\tt streams.*}, and {\tt stream\_list.*} files to the run directory. 
	\item Edit the namelist files and the stream files appropriately (Chapter \ref{chap:running_mpas_a}).
	\item {\bf Optionally,} prepare meshes for the simulation (Chapter \ref{chap:mpas_grid_preparation}).
	\item Run {\tt init\_atmosphere\_model} to create initial conditions, then run {\tt atmosphere\_model} to perform model integration.
	\item Visualize output, and perform analyses.
\end{enumerate}
