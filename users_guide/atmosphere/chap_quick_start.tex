\chapter{MPAS-Atmosphere Quick Start Guide}
\label{chap:quick_start}

This chapter provides MPAS-Atmosphere users with a quick start description of
how to build and run the model. It is meant merely as a brief overview of the
process, while the more detailed descriptions of each step are provided in later
chapters.

In general, the build process follows the following steps.

\begin{enumerate}
	\item Build MPI Layer (OpenMPI, MVAPICH2, etc; Section \ref{build_prerequisites})
	\item Build serial NetCDF library (v3.6.3, v4.1.3, etc; Section \ref{serial_netcdf})
	\item Build Parallel-NetCDF library (v1.2.1, v1.3.0, etc; Section \ref{parallel_netcdf})
	\item Build Parallel I/O library (v1.4.1, v1.6.1, etc; Section \ref{pio_build})
	\item Build METIS library and executables (v4.0, v5.0.2, etc; Section \ref{sec:metis})
	\item Checkout MPAS-Atmosphere from repository
	\item Build init\_atmosphere and atmosphere cores; Section \ref{compiling_MPAS}
\end{enumerate}

After completing these steps, executable files named {\tt init\_atmosphere\_model} and
{\tt atmosphere\_model} should have been created in the top-level MPAS directory. Once
both executables have been created, one can begin the run process as follows:

\begin{enumerate}
	\item Create run directory.
	\item Copy executables to run directory.
	\item Copy namelist.input, input and graph files into run directory. 
	\item Edit namelist.input to have the proper parameters.
	\item (Optional) Create graph files, using METIS executable (pmetis or gpmetis depending on version).  A graph file is required for each processor count you want to use.
	\item Run MPAS-Atmosphere (e.g.\ {\tt mpirun -np 8 atmosphere\_model}).
	\item Visualize output, and perform analyses.
\end{enumerate}
