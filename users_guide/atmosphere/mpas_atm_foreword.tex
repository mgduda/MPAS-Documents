\chapter*{Foreword}
\label{chap:foreword}

This user's guide describes the Model for Prediction Across Scales -- Atmosphere
(MPAS-A) Version \version. Updates to MPAS-A, including the most recent code,
user's guide, and test cases, may be found at \url{http://mpas-dev.github.com}.

MPAS-A is the non-hydrostatic atmosphere model built within a general MPAS
framework, which is being developed collaboratively between Los Alamos National
Laboratory (LANL) and the National Center for Atmospheric Research (NCAR).
Common functionality required by different MPAS models, such as parallel
input/output, time management, block decomposition, etc., is provided by the
MPAS framework, while development of specific model ``cores'' is handled by the
individual groups.

\vspace{8pt}
\noindent
{\bf Contributors to this guide:}\\
Michael Duda, Laura Fowler, Bill Skamarock, and Conrad Roesch\\
{\bf Additional contributors to the compilation chapter and appendices:}\\
Doug Jacobsen, Todd Ringler

\vspace{8pt}
\noindent
{\it The National Center for Atmospheric Research (NCAR) is operated by the
University Corporation for Atmospheric Research (UCAR) and is sponsored by the
National Science Foundation.  Any opinions, findings, conclusions, or
recommendations expressed in this publication are those of the authors and do
not necessarily reflect the views of the National Science Foundation.}
