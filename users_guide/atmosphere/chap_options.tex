\chapter{Model Options}
\label{chap:mpas_options}

Beyond the basic process of running a global simulation with standard output files outlined
in Chapter \ref{chap:running_mpas_a}, the MPAS-Atmosphere model provides several options
that can be described in terms of variations on the basic simulation workflow. In the sections
that follow, major model options are described in terms of the deviation from basic global
simulation process.


\section{Periodic SST and Sea-ice Updates}
\label{sec:sst_update}

The stand-alone MPAS-Atmosphere model is not coupled to fully prognostic ocean or sea-ice models,
and accordingly, the model SST and sea-ice fraction fields will in general not change over
the course of a simulation. For simulations shorter than a few days, invariant SST and sea-ice
fraction fields will generally not be problematic. However, for longer model simulations, it is
generally recommended to periodically update the SST and sea-ice fields from an external file.

The surface data to be used for periodic SST and sea-ice updates could originate from any number
of sources, though the most straightforward way to obtain a dataset in a usable format is to process
GRIB data (e.g., GFS GRIB data) with the {\em ungrib} program of the WRF model's pre-processing system(WPS).
Detailed instructions for building and running the WPS, and the process of generating intermediate data
files from GFS data, can be found in Chapter 3 of the WRF User Guide:
\url{http://www2.mmm.ucar.edu/wrf/users/docs/user_guide_v4/v4.1/users_guide_chap3.html}.

The following steps summarize the generation of an SST and sea-ice update file, {\tt surface.nc}, using
the {\tt init\_atmosphere\_model} program:

\begin{itemize}
\item Include surface data intermediate files in the working directory
\item Include a {\tt static.nc} file in the working directory (Section \ref{sec:atm_real_static})
\item If running in parallel, include a {\tt graph.info.part.*} in the working directory (Section \ref{sec:metis})
\item Edit the {\tt namelist.init\_atmosphere} configuration file (see below)
\item Edit the {\tt streams.init\_atmosphere} I/O configuration file (described below)
\item Run {\tt init\_atmosphere\_model} to create {\tt surface.nc}
\end{itemize}


\begin{longtable}{p{3.0in} |p{3.25in}}

\&nhyd\_model\\
   \namelist{inl:config_init_case}       = 8                      & must be 8, the surface field initialization case \\
   \namelist{inl:config_start_time}      = '2010-10-23\_00:00:00' & time to begin processing surface data \\
   \namelist{inl:config_stop_time}       = '2010-10-30\_00:00:00' & time to end processing surface data \\
/\\
\\
\&data\_sources\\
   \namelist{inl:config_sfc_prefix}      = 'SST'                  & the prefix of the intermediate data files containing SST and sea-ice \\
   \namelist{inl:config_fg_interval}     = 86400                  & interval between intermediate files to use for SST and sea-ice \\
/\\
\\
\\
\&preproc\_stages                                    & only the input\_sst and frac\_seaice stages \\
   \namelist{inl:config_static_interp}   = false                & should be enabled \\
   \namelist{inl:config_native_gwd_static} = false           & \\
   \namelist{inl:config_vertical_grid}   = false                & \\
   \namelist{inl:config_met_interp}      = false                & \\
   \namelist{inl:config_input_sst}       = true                 & \\
   \namelist{inl:config_frac_seaice}    = true                 & \\
/\\
\\
\&decomposition\\
   \namelist{inl:config_block_decomp_file_prefix} = 'graph.info.part.' & if running in parallel, needs to match the grid decomposition file prefix \\
/\\

\end{longtable}

After editing the {\tt namelist.init\_atmosphere} namelist file, the name of the static file, as well as the name of the surface update file to be created, must be set in the XML I/O configuration file, {\tt streams.init\_atmosphere}. Specifically, the {\tt filename\_template} attribute must be set to the name of the static file in the {\tt "input"} stream definition, and the {\tt filename\_template} attribute must be set to name of the surface update file to be created in the {\tt "surface"} stream definition. {\em Also, for the ``surface'' stream, ensure that the ``output\_interval'' attribute is set to the interval at which the surface intermediate files are provided.}


\section{Regional Simulation}
\label{sec:regional}

Describe changes needed to produce regional ICs and LBCs, and to have the model apply LBCs.


\section{Sounding Output}
\label{sec:soundings}

Describe use of `soundings.txt' file and namelist to obtain periodic soundings at specified locations.


\section{Incremental Analysis Update}
\label{sec:iau}

Describe operation of incremental analysis update.
